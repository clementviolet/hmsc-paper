\begin{enumerate}
\def\labelenumi{\arabic{enumi}.}
\item
  \emph{Joint Species Distribution Models} (\emph{jSDM}) are
  increasingly used to explain and predict biodiversity patterns.
  \emph{jSDMs} account for species co-occurrence patterns and can
  include phylogeny or functional traits to better capture the processes
  shaping communities. Yet, several factors may alter the
  interpretability and predictive ability of \emph{jSDMs} : missing
  abiotic predictors, omitting ecologically-important species, or
  increasing the number of model parameters by adding phylogeny and/or
  trait information.
\item
  We developed a novel framework to comprehensively assess the
  interpretability, explanatory and predictive power of \emph{jSDMs} at
  both species and community levels. We compared performances of four
  alternative model formulations : (1) a \emph{Benchmark} \emph{jSDM}
  with only abiotic predictors and residual co-occurrence patterns, (2)
  a \emph{jSDM} adding phylogeny to the \emph{Benchmark}, (3) a
  \emph{jSDM} adding traits to model 2, and (4) the \emph{Benchmark}
  \emph{jSDM} with additional non-target species that are not of direct
  interest but potentially interact with the target assemblage. Models
  were fitted on both presence/absence and abundance data for 99 target
  polychaete species sampled in two coastal habitats over 500km and 8
  years, along with information on 179 non-target species and traits
  data for the target species.
\item
  For both presence/absence and abundance data, explanatory power was
  good for all models but their interpretability and predictive power
  varied. Relative to the \emph{Benchmark} model, predictive errors on
  species abundances decreased by 95\% or 53\%, when including
  non-target species, or phylogeny, respectively. These differences
  across models relate to changes in both species-environment
  relationships and residual co-occurrence patterns. While considering
  trait data did not improve explanatory or predictive power, it
  facilitated interpretation of trait-mediated species response to
  environmental gradients.
\item
  This study demonstrates that any \emph{jSDM} formulation comes with
  tradeoffs between either explaining or predicting the occurrence or
  abundance of species. Hence, it highlights the need to compare
  alternative model formulations using the original and comprehensive
  assessment framework proposed in this study. Overall, this work
  contributes to a better understanding of \emph{jSDM}s' performances
  across multiple facets and provides insights and tools for model
  selection based on specific objectives and available data.
\end{enumerate}
